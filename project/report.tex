\documentclass[sigconf]{acmart}

\usepackage{hyperref}

\usepackage{endfloat}
\renewcommand{\efloatseparator}{\mbox{}} % no new page between figures

\usepackage{booktabs} % For formal tables

\settopmatter{printacmref=false} % Removes citation information below abstract
\renewcommand\footnotetextcopyrightpermission[1]{} % removes footnote with conference information in first column



\begin{document}
\title{Big Data and Its Application in Education}

\author{Weipeng Yang}
% \orcid{1234-5678-9012}
\affiliation{%
  \institution{School of Education, Indiana University Bloomington}
  \streetaddress{201 N Rose Ave}
  \city{Bloomington} 
  \state{Indiana} 
  \postcode{47405}
}
\email{yang306@umail.iu.edu}

\author{Geng Niu}
% \orcid{1234-5678-9012}
\affiliation{%
  \institution{School of Education, Indiana University Bloomington}
  \streetaddress{201 N Rose Ave}
  \city{Bloomington} 
  \state{Indiana} 
  \postcode{47405}
}
\email{Niugeng@umail.iu.edu}

% The default list of authors is too long for headers}
% \renewcommand{\shortauthors}{Yang. W, Geng. N}


\begin{abstract}
The development of big data is changing the society in a dynamic way. With high speed internet and high penetration rate of mobile devices, every individual becomes a source of data and is constantly provide these data to various organizations who want to make profits or want to utilize these data to contribute to a certain end. Big data helps online retailers to add new strategies to increase their selling; it also helps medicare organizations to make more accurate diagnosis to patients;it even changes the sports industry. Because big data is changes various industries profoundly, it will certainly in a point change the way people acquire new knowledge. For this reason, it is important to review how big data bring changes to different industries and how education strategies could adjust to the fast-change world. In order to have a clear picture of in what ways big data will influence educational strategies, we will use education recommendation system and medicare education which is transformed by big data, as a case to see how educators should adjust their strategies with the benefits brought by big data..
\end{abstract}

\keywords{i523, HID236,HID218, big data, education}

\maketitle

%here begins the body of the document






\section{Introduction}
When we search the definition of education, we will find that it is defined as process of facilitating learning or the acquisition of knowledge, skills, values and beliefs and habits \cite{Wikipedia2017}. However, when we talk about education, we regard education as academic education such as K-12 education and higher education. In reality, education is happening all the time and everywhere. When you are sitting in a classroom, you are learning to get a degree and plan for your career life. When you are working in the office, you gradually learn how to accommodate what you learn at school to what the reality is in the workplace. When you are watching TV, you are getting information about what is happening around the world and you gain your first impression of different countries around the globe. And even when you are shopping online, you are learning how to identify the product is good or bad by reading the reviews. Therefore, in the 21st century when educators look into education, it will cover not only school education but also corporate training and other forms of informal learning. 

\section{A look back at learning}
It is very difficult not to think of Confucius and Socrates, two great men who are regarded as the most important people in terms of influence on education in the East and West. Confucius emphasized the importance of education and proposed that education should be equal to everyone. He was a teacher himself and taught students morality, proper speech, government and refined arts. ``He never discourses at length on a subject. Instead he poses questions, cites passages from the classics, or uses apt analogies, and waits for his students to arrive at the right answers'' \cite{Riegel2013}. Confucius set an example of teachers in ancient China and other regions in Asia such Korea and Japan. Teachers in these parts of world should be much better learned than students and are examples in terms of morals. Because of this, students should imitate teachers. By contrast, Socrates adopted a different view of learning. “Socrates does not believe that any one person or any one school of thought is authoratative or has the wisdom to teach  ``things.'' Socrates repeatedly disavows his own knowledge and his own methods. However, this appears to be a technique for engaging others and empowering the conversator to openly Dialogue \cite{Burgess2011}. This may be one of the reason why the learning style in the East and the West is so different. But when we look back at education in schools in the past, no matter where the school or the learning place is located, learning is highly teacher-centered. In ancient China, teachers still follow the learning style of confucius. The teachers had authority over their students and students were supposed to treat their teachers like their “fathers”. In return, teachers should be selfless enough to pass what they knew to their students. In the West, the situation was similar. And one of the reasons might be the lack of resources in teaching. In ancient times, only limited number of books were available due to printing technique and the number of scholars who could write books. In the year of 1500, the illiteracy rate of men and women in English is 90 and 100 respectively. And in Qing Dynasty around 1880, literacy rate is around 30-45 in men and 2-10 in women. It is not hard to deduce that these numbers are much lower in 1500s \cite{RaWski1979}. Knowledge or information was in the hand of the top 10 of people which make knowledge more precious. Therefore, teaching must be teacher-centered and teachers have much authority over the students. However, it is never true today. The volume of books will take a person's whole life to read; TV has changed the way of how people get information; and the internet revolutionizes how information is created and transmitted.  As a result, teachers are no longer just knowledge provides and it is impossible for them to be mere knowledge providers because learners have access to almost infinite source of knowledge.

\section{What is instructional design?}

``Educational technology is the study and ethical practice of facilitating learning and improving performance by creating, using, and managing appropriate technological processes and resources'' \cite{Januszewski2008}.  Instructional technology started from instructional media , the use of which can date back to the first 10 years of 20th century in school museums. The use of different media in instruction or learning have gone through visual instruction, audio-visual instruction and the use of communication theories to today the integration of computers and internet technologies \cite{Reiser2012}.

However, the turning point of the birth of educational technology began as visual education. At the turn of the 20th century, educators were exploring the potentials of motion pictures and projected slides. In the 1950s, the advent of television added new dimension of widespread of audio-visual programming. At this time, the design materials only focused on creating attractive and creative presentations which are pleasing to learners' eyer and ears. But a shift happened in the next decade. Educators not only cared about the appeal of the teaching or learning materials, but also cared about what learners are doing. In the next a few decades, the focus of learning design continued to change because of the advent of the internet which allows learners to collaborate anytime anywhere. Also, computers became a powerful assistant in learning with the advances made in CPU and storage \cite{Molenda2008}.

Instructional design has several names such as instructional system technologies, learning design and educational technology. Although universities which have program on instructional design prefer different names, they have similar courses and goals of training. The definition of instructional design has been revised several times in history and those changes were caused by different opinions held by experts in the field and most important caused by advances of science and technology.

\section{The development of education psychology}

Behaviorists believe that performance of people can be changed by contingencies of reinforcement combined with changes in the environment \cite{Skinner1954}. For example, drivers or passengers of a car may not want to or forget to fasten the safety belts. In order to prevent that from happening, a machine would give out a loud ``beep'' noise which is annoying to tell people in the car that you need to fasten the belt. This is called a negative reinforcement. In order to avoid something awful to occur, people will behave in certain way. Opposite to this is positive reinforcement. For instance, teachers would give a student who has the high scores in a exam a gift and verbal appraise as a way to encourage all the students to study harder. Behaviorism has great impact on programmed instruction in both academic education and military training \cite{Saettler1990c}. However, the behaviorist approach to learning has two problems. The first one is the use of proper reinforcement. As learners grow older, instructors have to find reinforcement that learners will value. But it is really difficult to provide such reinforcement to adult learners. The other problem is that behaviorists seem to ignore the process of learning.

By contrast, cognitivists focus on how people process information. The core components of cognitive approach to learning are perception and sensory stores, short-term memory, and long-term memory. Perception is about how people select what information to pay attention to; ``sensory stores are capable of storing almost complete records of what we attend to but hold those records very briefly'' ; short-term memory helps people to rehearse the information coming through sensory stores but it has limited capacity ; Long-term memory is where information is stored in a certain way permanently and is ready to be retrieved \cite{Silber2006}.  A example of using cognitive theory in learning or information is the design of presentation slides. A good PowerPoint presentation may contains clear contrast between the texts and the background or different categories of information. It also only contains keywords of a topic so viewers will be well-guided by the slides when listening to the speakers. Although cognitive approach of learning helps learners to process information, what learners can do if they need to learn a ill-defined topic?

Constructivism made one more step forward towards learning.  Learning, according to constructivist theory, is a process of meaning making, a process of solving problems when encountering cognitive conflict and a social activity such as collaboration and negotiation \cite{Wilson2012} Put it simply, constructive theory advocates simulation of the real world environment. A topic is ill-defined and learners are required to formulate their own strategies to look for relevant information and experiment potential solutions to solve a problem.

\section{Problem-based learning and medical education:}
Problem-based learning is based on constructive approach to learning. ``Participation in valued activities within different domains is fundamental to how students learn.'' People who advocate this problem-based approach suggest that learning happens when other people involve such peers, tutors or mentors. And cooperation in activities can lead to higher reasoning level. Students may change their perspective of thinking and their opinions about a topic because in collaboration more ideas are involved and those ideas will be discussed in an environment in which sharing and collaboration are promoted. Lev Vygotsky found the construct of the zone of proximal development to explain how people can facilitate knowledge construction. This framework shows that if instructors can reduce the distance between what the learners can do completely by themselves and the things that can be accomplished by themselves with the assistance from others, then the instruction can be successful. ``PBL is a form of education in which information is mastered in the same context in which it will be used'' \cite{Donner1993}. Another definition of PBL is ``a learning method based on the principleof using problems as a starting point for the acquisition and integration of new knowledge''\cite{DIMansur2012}. Problem-based learning can also refers to problem-and -task-centered approaches of learning. It is one of educational technologies designed to situate instruction in authentic or meaningful settings. It has been employed in many different fields of studies such as medicine, science, law, business and mathematics. And the goals of PBL differs from each other. In medicine PBL requires learners to work in groups to practice their skills to diagnose patient cases and the ability to use clinical knowledge in practice. But in science and humanities, students in a PBL class to come up with explanations for a certain phenomenon through activities such as defining a question, seeking evidence, and outlining and argument. Moreover, in law and business related courses students will engage in the study of cases and they will be encouraged to seek and summarize critical information from those cases and present their finds to peers in the classroom. By the end of their presentation, instructors will provide feedback. In math and science PBL courses, students work together in an environment in which constructive feedback is provided to each other or by tutors or teachers. Although the goals of learning in different fields are different, in PBL the core value is to put education in authentic tasks so the learning is more meaningful.\\


Savery and Duffy proposed a framework of how to conduct problem-based learning:
Anchor all learning activities to a larger task or problem\\
Support the learner in developing ownership for the overall problem or task\\
Design an authentic task\\
Design the task and the learning environment to reflect the complexity of the environment they should be able to function in at the end of learning\\
Give the learner ownership of the process used to develop a solution\\
Design the learning environment to support and challenge the learner's thinking\\
Encourage testing ideas against alternative views and alternative contexts\\
Provide opportunity for and support reflection on both the content learned and the learning process \cite{Savery2001}

Medical education is very suitable for Problem-based learning because the advances made in medicare makes it impossible to include everything in lectures. And in the field of medicare, doctors will face various problems with patients which are highly likely beyond what medical students can learn from school. Therefore, in order to foster the ability to solve problems, critical thinking and experiment potential solutions, PBL serves as a critical part of medical education. PBL has been employed in many medical schools around the world. ``It was introduced in the medical school at Mc-Master University in Canada in the late 1960s and is now a common curriculum component in medical and health science schools around the world'' \cite{DIMansur2012}. ``The University of New Mexico was the first to adopt a medical PBL curriculum in the United States and Mercer University School of Medicine in Georgia was the first U.S medical school to employ PBL as its only curricular offering'' \cite{Donner1993}.

Here we will present how to use this framework with medical education. 
The first step of designing PBL for medical students is to find an authentic task. By saying authentic task, we don’t mean the task must be the same with what happens in a hospital every day. It means the task will require similar cognitive load to a real problem. The specific difficulty of the task is designed by the instructor according to the level of the course. As is often the case, instructors will create scenarios to represent a authentic task. Before creating a scenario, instructors should formulate objectives of the course, and create a scenario in which all of these objectives will be accomplished. The complexity or the difficulty of the problem should be appropriate to the curriculum and the level of students' understanding. It is better if the scenario is appealing enough to attract students' attention. Basic science should be included in the context of a clinical scenario to encourage integration of knowledge. Although the problem presented in a PBL class should be ill-defined, the PBL scenario should have cues to stimulate discussion and push students to seek reasonable explanation to the issues involved in the scenario. At last, the scenarios should promote participation by the students in the seek of explanation \cite{Wood2003}.\\


The next step is to gather all relevant information which can be useful or useless to the final solution to the diagnosis. However, this information will not be provided to students directly. Besides relevant information, the instructor should also have other resources such as equipments, lab or simulation of the environment in a real hospital.
Then students will have a meeting with the instructor to talk about the basic information of the patient such as age and gender and symptoms he or she has. Then the task will almost completely hand to the students. Here comes to an important point of the whole learning experience. The learners should be told that there is no correct answer to the task and it is the students' responsibility to find a possible solutions.

A tutor should be assigned to the students and the tutor is not necessarily an expert in medicare because the tutor's job is not to provide suggestions to the students. Or the instructor can be the tutor. But the responsibility of the tutor is to ask leading questions such as why do you choose this? or how did this happen? By asking these questions we hope students will spot their own mistakes or loopholes in the process of finding a solution to the diagnosis. In this way students will revise their strategy of work. Another responsibility of the tutor or instructor is the provide key information if the students are seriously off the track. And in order to ensure the quality of the course, tutors or instructors have to do so. Speaking of the roles of tutor in PBL, we have to talk about scaffolding. A real or physical scaffold is a structure to support learners to complete a task and it is not permanent. When a tasks is accomplished, the structure will be removed. It is still the same when we talk about scaffolding in education. It will be removed when it is not needed. Scaffolding is designed to assist learners to complete tasks which are otherwise beyond their reach. This suggest that the design of scaffolding must be very careful. So there are several questions for tutor and instructors to think when they design such a structure: what is needed to support, when and in what way to support the students, how much support should be provided to learners, and when and how to fade scaffolding.


In PBL in a medicare course, students are required to write reports weekly or bi-weekly on how they collaborate, what problems occur and how the solve this problems. Also, by knowing the progresses students make, it is much easier for instructors to see how they grow and how they should adjust some elements of the learning environment. And a final report to summarize the whole process of collaboration and the working process will be submitted at the end of the semester.

Such courses can also be conducted in multiple groups. Every group will have their own way of collaboration and propose different solutions to a diagnosis. And instructors should create an environment where different groups are eager to share their own progress because they can always get constructive feedback from their peers. Moreover, such a sharing environment will make the learning more dynamic and accelerate the growth of learners.  Tutorials are also an important element in PBL. Usually, the PBL tutorial have a group of students which has no more than 10 and a tutor who provide scaffolding to the session. The duration of the session varies. It depends on how long it takes for a certain group to have good dynamics.  Moreover, for each tutorial sessions , a different leader or chair should be elected so every member of the team will contribute and free ride can be avoided \cite{Wood2003}.\\

 Here we want to elaborate tools and activities can be included in PBL in the Internet era. Basically,  PBL courses can include activities such as generating lists, scaling down the scope of topics, making outlines of options, debating issues, and even voting. Today, many activities can happen in virtual environment.  Wikis enable learners to have meeting in a virtual community and collaborate on projects and solve problems. And meeting tools such as Zoom, Goggle Hangouts and Adobe Connect enable online meetings of a large group of students and share screens and notes. Moreover, blogs also provide virtual space for learners to practice their writing skills and share their writing with audiences beyond their teacher. In a PBL class, web 2.0 tools can also be included such as Skype, Twitter, Instagram.


Here are some examples of how Mercer University conducted PBL in its medical courses.  At Mercer University, a series of tutorial sessions were used to substitute the lectures. And during each session, faculty members and students would meet to discuss the actual case problems. In other programs which are related to clinical skills and community science, students need to deal with simulated patients and spend some afternoons with local primary-care practitioners. ``In this way, real life clinical practice in a rural community becomes a laboratory exercise for the illustration of basic science theory.'' In Mercer University, tutors were called ``faculty overseers'' who are neither to be the source of all information nor even to have information about every area being discussed. The responsibility of these overseers is to keep student participation and knows enough to prevent gross mistakes. On the contrary, students were teachers and learners. Without giving lists of what to know, students need to generate a list of what to look for according to importance of relevant information.\\

Although small groups of meeting played essential role in the PBL of Mercer University, lectures are still used to some extent. The students may have some lectures on one or two basic science lectures every week but these sessions were not mandatory. The evaluation of the course was intense. Students at Mercer University were tested by both intramural and extramural means. At the end of each the thirteen curricular phases, students would have  a 200-item, cross-disciplinary, objective examination and a forty minutes case analysis oral examination.

The majority of faculty members faver PBL over the conventional way of teaching. The reason is very simple: it is a more natural way of learning. PBL simulate the environment where people generate knowledge. For example, students became better prepared in the learning process.  That is the ownership was handed to the students instead of the instructors. If a student came to the discussion session without any preparation, she or he would be complaint by other members. Another benefits of PBL is that students became more flexible in learning. Students at Mercer University used texts, mono graphs, periodical literature and various resources in their learning. In the past, the learnng is very lecture centered and students were actually not actively engaged in the learning. By contrast, when they were on their own, they tried every alternatives to find useful resources and developed flexibility in learning \cite{DIMansur2012}.\\



From the history of the evolution of educational technology we can see the changes are brought by technological development made in other fields of studies. Those technologies were not intended to contribute to education but they are all utilized in education. And to successfully employ PBL in an academic learning environment, instructors and instructional designers must build a proper environment. As a result, the development of big data can provide new thoughts in how to advance current instructional design and improve the building of a proper PBL learning environment.\\

\section{Challenges of learning in the information era}

The challenges of learning in 21 first century is that the explosion of information brought too much information whose credibility is uncertain. Many people, especially scholars, questions the accuracy of  information of Wikipedia. However, Wikipedia may be the most popular sites for all kinds of information ranging from entertainment to academia. And because of the affordable and high speed internet, everyone has a say in the virtual world. One can find people argue on an issue in online forums, express their own opinions in blogs and social media. However, these information could be wrong and there is no third party to verify if the information is correct. By the end, people tend to believe in the opinions presented by the most popular sites or people. For example, in China a high school history teacher go visual on the internet and he starts to have his own online courses about history in China and other parts of the world. His courses are pretty interesting because a lot of humor is involved and various media are used such as animation and movies. Therefore, a lot of students prefer to watch his online courses instead of taking the face-to-face class at school. However, the opinions presented by this history teacher are very different from main-stream scholars especially in the history of the second world war and civil war in China. And this caused problems at schools. In this case I presented, the teacher actually unconsciously took advantage of the populism of teenagers at high school. Students at this age can be very disobedient and do not want to engage in the old tradition. 

And here comes another problem and the internet era. It is often the case that who has the most resources to populate a opinion will finally be the person who has the most say. It seems that the internet give people equal opportunity to express. However,  what really happens is that people can only find limited opinions or values. For examples, many news agency can use the resources they have to control media on what to be reported and what not to be reported. The fake news of several US news agency proves that it is real. In addition, the internet world is actually not so different from the physical reality. One is likely to find that the best resources on the internet are also expensive and only open to a few instead of the public general. That is also one of the reason why Wikipedia can be so popular because it is free to everyone. Because the best resources are only open to a small group of people that may widening the gap between the well-educated and the ill-educated. That is also the reason why the education community are working on Open Education Resources. Work with people from the academia, these open resources can be affordable or even completely free and still they have high quality. Massive Open Online Courses can be viewed as the most popular representatives of OER. However, there is still a long way to go in promoting education equality due to political and financial reasons.

Challenges to instructional design
The last challenge is how to do a thorough analysis of learning. In instructional design, ADDIE model is the most used model of doing the design process. ADDIE stands for analysis, design, development and evaluation. In the analysis phase, instructional designers need to work with subject-matter experts to formulate learning objectives. The learning objectives are specific performance which can be observed or evaluated in other ways. And learner analysis will include the traits of learners, the learning styles of learners, the motivation, confidence, prior knowledge of learners and the potential satisfaction of learners. Also a context analysis will also include. In the design phase, instructional designers will script and finalized learning strategies and tactics for the entire learning experience based on the analysis made in the first phase and the learning materials given by instructors. Then they enter the development phase in which the final education product is made. In the evaluation phase, instructional designers will conduct trials of the course and general a report on what needs to be modified and summative evaluation will be formulated to test learners' performance change in the end.

However, in the internet internet era, the number of online learners can be bigger than 2,000. In many popular MOOCs, there are more than 2,000 people registered. As a consequence, it is impossible to do a learner analysis. Not only the number of  students is big, the learning styles, motivations and level of prior knowledge vary drastically. Even if a comprehensive learner analysis is possible, the result might be that the learning environment is too complex and the course may be out of control of the instructors' hand. And in reality it is true. In many MOOC courses, learners have different expectations toward the same course, once they feel disappointed about the course, they drop. And the result is only a very small percentage of learners finally complete the course. And because of the huge number of students, the discussion forum goes out of control and instructors and teaching assistants cannot monitor the discussion and the discussion result in nothing. 

\section{How big data influence different industries}
Before we look at how big data will influence education or more specifically influence instructional design of medical courses in a PBL environment, we will first examine how big data have influenced other industries. The experience from these industries will provide guidance on how education community utilize big data.\\
Big data has become the buzz words for today's world. One of the reasons is that big data increase benefits of many business. The traditional way of costumer consumption has lasted for centuries. In the ancient time, people would go to fairs to buy groceries, hardware and clothes. But at that time, fairs were not standardized, and the conditions of those fair can be terrible. It was impossible to guarantee the quality of the goods bought by customers. Later, in the industrialized world, cities were built and shopping mall appeared. In a shopping mall, customers could buy good qualities in different stores. Instead, they would go to supermarket to buy groceries. This mode of doing business remained until the beginning of e-commerce. In the web 2.0 era, search engines enabled consumers to look for products in virtual shops and sellers can collect feedback of consumers' satisfaction in their website \cite{Chen2012}. Today, with big data technology, it is possible for online retailors to monitor activities of consumers online. Business owners can have better understanding of consumers and formulate more targeted strategies of how to increase profits. Because of the ability of monitoring online activities and better understanding behaviors of consumers, online retailers can provide personalized services. This is realized through the use of recommender system. Online buyers will be labelled according to their online activities and they will receive emails or suggestions of what to buy on the internet. 35 percent of Amazon's revenue is created by the recommendation system. Users of Amazon can click the recommendation section and see the products selected by the recommendation system. For example, if a learner is looking for a backpack, he or she will probably see some recommended backpack \cite{Krawiec2017}. Dynamic pricing is also a strategy brought by big data technology. ``Some business set different prices for their products or services based on algorithms that take into account competitor pricing, supply and demand and other external factors in the market. It is a common practice in industries such as hospitality, travel, entertainment, retail, electricity and public transport'' \cite{Wikipedia2017}\\

Before big data was brought to the face of the healthcare system, the role of data in the healing process of patients was minimal. Data such as name, age, disease description, diabetic profile, medical reports and family history of illness were collected. These data could only reflect limited view of a patient. For example, a doctor may know that the reason of a patient with heart disease can be traced back to his or her family, but there are many possible perspectives on why the patient has such disease.(Pal2016)
``The influence of big data on medicine is that we can build better health profiles and predictive models around individual patients so that we can better diagnose and treat disease.'' The pharmaceutical industry is facing the limitation of insufficient understanding of the biology if disease. But big data can help in building the understand of what constitutes a disease such as causes from DNA, proteins and metabolites to cells, tissues, organs, organisms, and ecosystems \cite{Schadt}.  The problem for the medical research is that enterprise is unable to follow the pace of the information needs of patients, clinicians, administrators and policy makers. “The flow of new knowledge is too slow, and its scope is too narrow.” The consequence of the medical research community not adopting big data technology is that hospitals are ill prepared for a more precise diagnosis. Now the medical research community need new thinking in their work. The new thinking must involve the integration of new technologies. ``For instance, researchers can use big data to reveal clusters of patient groups that might suggest new taxonomies of disease based on how similar they are according to a broad range of characteristics, including outcomes.'' Advances in prediction can simply attribute to the learning of data and creating a mechanism which is highly reproducible and has consistent performance \cite{Krumholz2014}.  ``Big data has helped healthcare institutions take a 360 degree view of a patient's health problems.''  With the help of big data, new findings, innovative methods of treatment plans and more precise diagnosis can be realized. Here is an example of how it is possible to build better health profiles. Some diseases are more common among a certain race of people due to genetical reasons. When a patient from this race is found suffering from heart disease, the doctors can look at the data of patients belonging to the same race who have same problems. By examining their life style, genetic structure, family DNA and other elements, they can build health profiles for these group of people. Wearable devices can also play a role in the detection of potential health problems even if no apparent symptoms are presented. Wearable devices can help see some indicators of health. And doctors can make certain conclusions and decide on the future action on them. The devices today are already  able to record data such as heart rate, pulse, glucose levels and calorie levels.  And  big data will also have the potential to personalize medicine. The NCI-MATCH trial is examine 1000 people who have tumors that do not respond to standard cancer treatments. Researchers hope that they can match drugs to this kind of tumor to produce the best result \cite{Pal2016}. `` In the very near future, you could also be sharing this data with your doctor who will use it as part of his or her diagnostic toolbox when you visit them with an ailment. Even if there's nothing wrong with you, access to huge, ever growing databases of information about the state of the health of the general public will allow problems to be spotted before they occur, and remedies - either medicinal or educational - to be prepared in advance'' \cite{Marr2015}.




\section{How big data will influence education in general}
The first change we will see in education is the rise of adaptive learning. Adaptive learning means that students can learning knowledge whose difficulty is suitable for their ability. This is enabled by the availability of online application, classroom activity software, social media, blogs and surveys of staff. With adaptive learning comes the universities' ability to provide personalized feedback to students, monitor student satisfaction, increase attainment and give students' opportunities to reflect on their own learning. On the other hand, instructors will  receive real-time reports which will enable them to adjust teaching strategies for the best outcomes \cite{Learning2016}.  Because learning is more adaptive, students can advance their learning in different paces. Big data and data analysts will inform instructors who is learning faster and can advance to a more difficult class and who need support from teachers \cite{Kerns2013}. 

Since learning of different learners will at different paces, it is important for learners to develop self-management. For example, in a PBL class, students need to solve an ill-defined problem and the process of learning is almost unguided. As a result, students need to take the initiatives and actively contribute to the project. Also learners will monitor their own process of learning and submit a report to summarize this process. So they must develop their meta-cognitive skills which means the learners are able to learn how to learn. Another reason why self-management is more important in the big data era is the widespread of informal learning. As mentioned before, people today are learning anytime anywhere. Social media, blogs, news and anything connected to the internet will serve as a source of learning. Therefore, it is impossible for teachers to monitor learning of students all the time.

\section{Big Data Mining}
Big data mining refers to the procedure in which a gigantic amount of data from a wide variety of source is collected, and analyzed with a wide spectrum of means to discover inner mechanism or other information via pattern \cite{Winoto2012}. Being used in almost every field such as business marketing, science and engineering, medicine, design and education industries to provide such functions as intelligence, research and marketing. Oftentimes, big data mining will be carried out on individual persons. When someone is doing activities online, their data will be collected. They could also be providing these data via questionnaires, surveys or other means. This massive amount of data collected on everyone are commonly called big data by the industry and corporations and companies will utilize them to figure out what need one have or what kind of personal trait one may carry.
As the big data industry found itself in rapid development, concerns and other critiques are also rising on the ethical issue of big data. Heated discussions were talking about the insult to privacy and abuse of such data. However, big data have already set foot in so many industries and almost all aspects of our daily lives\cite{Siemens2013}.\\
Data mining sees Artificial Intelligence and Machine Learning as its inception. During data mining, patterns are discovered, and data scientists could utilize such patterns to carry out more versatile functions. The system could get to understand an individual via the data collected about this one. The Recommendation Engine, or so called the Recommender System, is one application for data mining. The recommendation engine filters information and uses data mining techniques to figure out the specific suggestion to one person for information or other assets that may help them with current or future needs.\\
One commonly used example of the recommender system would be the online shopping websites. When someone shops on it, he or she will be given information about merchandise that related to this purchase. Such recommendations require various kinds of variables, such as this person's shopping history, the gender, age, and occupation of the person, or the items other bought after purchasing the same item. Another example will be after someone searched for a merchandise or service online, the advertisement will pop out for them showing related products.\\
These kinds of systems require algorithm with high complexity to give out recommendations following patterns discovered via enormous amount of data from mining. Such presentation will be oftentimes beneficial to individuals as they no longer need to go through such amount of information to find their desired service or product. Instead, targeted recommendation will be directly presented to the individual and sometimes the individual will have little awareness that they may need such product or service. As a result, the system will greatly enhance the efficiency for the user, it will also be a blessing for the services or products so that they could be utilized more often.\\
With such benefits, the recommendation engine is wildly used amongst all online websites, including but not limited to online shopping, searching, streaming and social media websites\cite{Shum2012}.
\subsection{Types of Recommendation Engines}
A good deal of recommendation engines is backed with such technique called as collaborative or content-based filtering technologies.\\ Collaborative filtering resembles a person making purchases on the gathered information from other via verbal or other means. It could also be understood as crowdsourcing\cite{Schrum2009}. In many online websites, people could give out ratings or feedbacks for others to reference. It is an interesting phenomenon that customers will more likely to read crowdsourcing comment first rather than the information provided by the seller. The collaborative filtering based system took one step further by categorizing commenters into different subcategories and present different person with different information or resources that might only be beneficial to him or her. Such patterns as statistical models are utilized to calculate everyone's correlation, thus giving out a value of recommendation. Some examples might be Twitter, eBay, Steam and Apple Store. They are all using collaborative filtering systems.\\
On the other hand, content based systems focus on different properties of a resource, in comparison with the properties of a person. As a person's total using time accumulates, the system will become more and more accurate as the user will demonstrate more personal traits and preferences in using the system. Examples of this kind of content based system will be Netflix and other streaming websites. Moreover, a developed recommendation engine could involve both collaborative and content based filtering techniques to bring prediction accuracy to a new level.
\subsection{Math Models of the Recommendation Engine}
The math models that standing in the back of the data mining engines include such technologies as association, classification and clustering means. Clustering refers to the procedure of combining individual with certain characteristics and trait being recognized as high value in the recommendation system. Such values as ratings, tones of comments are taken weighted average of all members in the cluster to identify the how the individuals in this cluster would recommend this product or service. More complicated systems would involve multiple clusters and calculated overall weighted averages across all the clusters that one individual belongs to\cite{Schafer2005}.\\
 Classification identifying technique are also utilized as the cornerstone of interconnecting different person with different appreciation to different items. Fundamental version of classification systems only works as primary filter to figure out how relevant individuals wit h desired kind of resources. As an example, only providing infant nutrition food for those who just give birth to a baby. This example only provides a crude vision of classification while more complicated ones will be able to perform prediction recommendations with higher complexity, and thus higher accuracy\cite{Rountree2005}.\\
Association on the other hand provide more sophisticated recommendation rules with the introduction of correlation amongst different items or different individuals.  With such rules, the system will be granted the ability to determine what a person needs most currently, rather than giving recommendations based on the person's previous activity history in the system. One example will be that if someone is looking for an oven in the kitchen, but he or she was browsing a dishwasher 2 months ago, the system will begin to give recommendations on oven or other cooking utensils, rather than kitchen cleaning utensils. Like mentioned before, a more complicated version of the association rule will give out recommendation with more complicated consideration and calculations. The recommendation system will be referencing different traits of a person or by viewing at a variety of items being browsed in the system by the user. One supplementary of the association recommendation system will be using dynamic analyzing to provide recommendations for future use when the user wished to need some resources that related to the current inquiry. An example would be a person who bought or browsed an oven today may be provided information of recommendation on oven recipe, or aluminum foil tomorrow.\\
\subsection{Big Data Recommendation Engines in Education}
As we have mentioned before, recommendation engines based on data mining are proving to be beneficial to almost all fields in our lives, and education is one of them.\\
The field in education that involves big data mining are often referred as learning analytics. It focuses on how big data mining could be utilized for teaching and learning purposes ranging from personalized teaching, learning, evaluations and assessments for individuals to providing data to decision makers of various levels of education (for example, a director of a department or a government official of education).  Big data mining has provided benefits to many aspects of education such as teaching, learning, education leadership, adult education, special education, enrollment decision, talent education, etc. The new millennium has seen the rapid development of educational big data mining and the field is hunger for talents that possess not only profound understanding in educational theory, but also the capability to carry out statistics, research and evaluation in education\cite{Romero2009}.\\
As the examples mentions above, the educational recommendation systems have deep similarities with commercial recommendation systems as they both strive to introduces the user to their desired products or services. However, educational recommendation system could also provide interconnection between learners, their desired course, their personal traits and educational resources that could serve the learners to help them reaching maximum efficiency in learning and to reach their academic goals. These beneficial factors make the educational recommendation engines a state-of-the-art asset for students to excel in personalized online learning systems. In such system, learner's characteristics, track selection and knowledge gained in previous learning could all be quantized into values to serve as a filtering and weighing standard to learners in e-learning. As one can see, such system has great flexibility and are highly adaptive to different learners. In this way, the efficiency of learning is greatly enhanced, and students are more motivated in engaging in learning\cite{Romero2008}.\\
The history of the e-learning recommendation system could be traced back to computer assisted instruction systems, also known as CAI. One major concept called Time-shared, Interactive, Computer-Controlled, Information Television (TICCIT) was invented in the last 70s. This could be the cornerstone of nowadays educational recommendation engine. TICCIT is developed so that the learner could have higher control in their own learning with the help of a mentor giving suggestions and advices from time to time. The education recommendation has met its rapid development afterward ever after the introduction of TICCIT as they could provide personalized advice and suggestions to learners according to their daily usage and browsing history of the system. Students could spend less time on looking for the education resources on their own or filtering out valued teaching and learning resources from a gigantic amount of information on the internet or within the e-learning system. In such way they could devote all their valuable time to learning, rather than being in a frustrated state without guidance\cite{Maloy2014}.\\
As mention before, the recommendation system's ability to provide an accurate result relies on massive amount of data collected from individuals and their behavior on the e-learning website. In this way, e-learning websites with a considerable number of users could better contribute to the learning process of the recommendation engine. For instance, Massive Open Online Courses (MOOCs) could have hundreds or thousands of active users on the website, or even learning the same course at one time. In such way, the recommendation engine evolves quickly, and user could benefit from it. Moreover, online learning websites have a social learning ecosystem which have great resemblance to social media networks. This lays the groundwork for the recommendation system to make full utilization of its huge user database to provide more relevant courses for learners. It is worth mentioning that these e-learning systems with social element are more likely to be involved with informal or professional learning. An example would be info of a user in career development system will be put under comparison to his or her colleague's information to carry out a performance evaluation\cite{Kingsley2011}.\\ 
Also, when he or she wishes to visit some of the resources on the career training website, the system could filter out his or her colleague's recommendation, comment, rate of one course and then utilize algorithm to provide this user with resources not only capable of helping him or her reaching current goals, but also courses and information that may become useful in the future\cite{lucas2012}.
Lots of educational teaching and learning systems with data mining and recommendation feature are established on online learning systems that could be easily visited from a mobile phone or a tablet. Such convenience no doubt made collecting data at great ease and allows more users to participate in such process. This could be a beneficial cycle: the ever-growing user base allows the algorithm to be more accurate and provide more personalized learn guidelines, and such feature will not only attract more user but will also let remaining users to provide more data to the system.\\
Learning analytics could also find itself useful other than the scenario of teaching and learning systems. Data mining and recommendation engines could be also used in supporting students in daily learning. For instance, a system that feature in college application could use a recommendation engine to provide learning track for students to better prepare for a certain university's requirements. Such method could be also used with other kinds of online learning motivation techniques such as badges. Badges are like achievement system in which when a student accomplish certain goals, he or she will be award a badge. He or she could get to know the global percentage of student holding that badge and get motivated in making more accomplishments\cite{lankshear2006}.\\
One more application would be the student retention system, in which students' data are monitored and a baseline is set based on the overall performance of all the student within. If one student's performance is below par, the system will receive alert and will send support or intervention staff as soon as they can to help the student and prevent him or her from dropping the course. In addition, big data also provide new thinking on how to conduct the PBL learning process. For example, when learners are working in the virtual environment, tutors can monitor the contribution of students in a certain group. In this way, the tutor can quickly identify who is not contributing to the team and take certain measures to intervene the performance of this student. Moreover, in a conventional PBL environment, the timing of proving scaffolding and removing scaffolding is very hard to master. But with the help of big data, tutors can analyze the process of learning in a team and spot the time when the team make minimal progress\cite{Kim2013}. In this way, the instructors and tutors can provide in-time support. And tutors can also spot the time when a team have sufficient knowledge and ability of accomplishing the task, so tutors can fade scaffolding. From the learners' perspective, big data give them space to try new ideas. Instead of having group discussions and debates of different ideas, students can also learn from what the data tells them and gain empirical experience. With big data technology, learners can also formulate more up-to-date solutions to a task\cite{Krumholz2014}.\\

Big data also provide powerful tools to instructional designers. With the help of data, instructional designers can label learners just as the way online retailers label customers. Then those labels will be put into different categorizes. This is very important for conducting learner analysis. Instructional designers will be able to see clearly the motivations they have, the prior knowledge they possess to determine the scope of learning. Also with such data, instructional designers can design proper strategies to motivate learners and increase the satisfaction rate. The learning style data will help the development of teaching strategies. Designers and subject-matter experts can integrate different ways of learning in one semester based course and let learners with different learning habits to collaborate to foster flexibility in learning.
\\
\section{Web analysis} 
Web analysis is also an uprising branch that belongs to the learning analytics and being supported by big data mining. It focuses on how to collect and analyze data gained on websites or applications that needs to connect to internet before using. These kinds of data are often a result of user's activities on the internet. Web analytics are often utilized to boost the study and learning efficiency of students in a specific Learning Management System(LMS). It could also help the administrator of the website to monitor and support student's learning progress and to help oversee the functioning of the website\cite{Krawiec2017}.\\
Web analytics are also utilized by administrators to get a better understanding of what kinds of personal traits one user may carry and how would this one interacts with various function on the website. It could be utilized to make a prediction on what kind of educational products or courses will be more welcomed by certain students and learners. After analyzing such data as how many people have visited one page and what kind of activities they are most likely to carry out, the administrator could be informed that what kind of needs one student possesses and how they can develop in the future to cater to their needs. For the education website owners, the web analysis can also be used as a mean to find out any hidden security risks online and could help them gaining evidence for court should an attack really happens. \\
In the world of academics, web analytics is also beneficial in helping the college to make strategic plans. For instance, with a growing number of traditional courses, tutoring services are going to be changed into their online version, web analytic will find out how to deploy these courses and servers better so that they could get maximum visit from those who are interested in them. Since many data and information are distributed on different websites, they call for the facilitation of web analytics to perform an integration to the scattered resources. Moreover, the educational corporations, both online and offline, could easily get to know how the traffic flow changes every day on their online learning systems\cite{Aher2013}.\\
\subsection{Web Analytics Anatomy}
Normally, the history of web analytics could be traced back to last 60s when scientists start to analyze web logs. These logs could be transaction or search types. The transaction type takes direct actions such as user's clicks, how long they have spent on one page into consideration while search type focuses more on the behavior on how the users carry out the searching activities. \\
Depending on the data provided by the servers, web analytics send small package of data (commonly known as cookies) to the user. Cookies will start collecting data and send them back to the server. Such process is called as server-side data collection. On the other hand, this kind of data collection would render itself not accurate. Internet service provider (ISP) provides IP address to users while user many set blockade to some cookies. To the contrary, client-side data collection is more flexible and can be more accurate. By implanting tags into the website being visited by the client, client-side data collection could carry out more versatile missions.\\
The word Human Computer Interaction(HCI) have been a buzzword nowadays and it have been embedded into our daily lives. Web analytics could also have utilized such different methods as interviews, questionnaires to establish more convincing reports. Key Performance Indicators (KPI) are set up to differentiate various kinds of web analytics. As of an example, one university that introduced with a new kind of LMS are facing difficulty because too much people are using its social features and it needs some backup support. The web based analysis will be performed to assess the resource the university possesses and evaluate the need to figure how to employ capable person to perform certain kinds of maintenance work as well. The KPI within could be able to indicate how many clicks can one user click before reaching the help page, how long will a user spend on the help page, how easy the help material could be comprehended, how visible are the various icons to the viewer and so on. Then the KPIs are collected and analyzed to compose a report\cite{Chughtai2013}.\\
\subsection{Web Analytics and Education}
In the field of education, web analytics are utilized to form reports that are driven by data to help such functions as facilitating students, managing staffs and supporting researches. Also, web analytics are used to figure out how well a student could perform, how would the student and online tutor would normally interact, how effective one course could be and how well the student is progressing in the course. One specific kind of web analysis is called as academic analytics. It would evaluate the overall performance on an online teaching website to provide information so that administrators could better make decisions. \\
Learning analytics, as mentioned before, focuses on the collection and analysis of data that have relation to the learners and the courses and learning materials. Learning analytics are also utilized in documenting students' overall learning efficiency in computer facilitated learning and could facilitate student to get accustomed to the online teaching and learning environment better. 
With big data gained and stored in the systems, learning analytics made many contributions to lots of fields that could help students reaching their academic success. For example, they could note down where the students are now in the middle of a course; if a student misses too much class, they could figure it out and send intervene staff quickly; assess different aspect of the Learning management system; give out help and facilitation that could cater to a student's need\cite{clifton2008}. \\
As one way to lead students to academic success, learning analytics could trace all students' activity and other behaviors in the online environment, with data collected via the student information system (SIS). In a class that is one hundred percent online, instructors could fully utilize learning analytic to carry out formative evaluation, which could help the teacher to learn about how the students are performing, how could they make modification to ongoing courses, and how he or she could demonstrate such course materials to the students. An example would be that the teacher could track how many time the student have entering the LMS in the allotted time and use it as evidence of attendance record. Also, the teacher could record how many clicks are carried out in one content page or during one course, or how long the student has spent in different sections of the course. All the data collected above could give out information on how the user behave and how the relations of learners and teaching and learning materials have been. With the deployment of web learning analytic in the LMS, the instructor could figure out abnormal activities of students and give out interventions that cater to the student’s needs. \\
To boost the efficiency of the learning analysis system to the maximum level, learning analytic could also reach to qualitative data such as the discussions in students' forums, students' cooperative wiki pages, and many other social learning assets to form more persuasive and convincing data to website administrators. This requires natural language processing kit (NLTK) to perform semantic analysis so that these qualitative data could be better transcribed into data that would be better analyzed. These data gained could be utilized to perform some higher-level assessments, such as the creativity and critical thinking level of a student\cite{Dwivedi2013}.
With different teaching and learning goals, KPI could help student, or make modifications of online course in the online learning website with the facilitation of quantitative and qualitative data. They can also run course diagnose for learners and teachers. The KPI mentioned could be collected and analyzed with such techniques as students' characteristics and performance tracking, investigating a group of students with same traits, giving out content recommendations of learning materials on history activities and make prediction to future developments.\\
\section{The Intelligent Tutoring Systems}
The term intelligent tutoring system are used to describe a computer system that could act as a human mentor to some extent to facilitate a student in getting to understand and have firm understanding of the learning materials. Such system is often designed to make learning with a higher efficiency as well as providing inspiration to students while learning. It requires the support of big data and are considered one of the rising learning technologies in the field\cite{Office2013}. \\
Taking a human mentor for example, he or she will be preparing for the learning material for the student first, then he or she will try the best to get the student motivated for learning. When the student is facing difficulties in learning, he or she will stand out and provide necessary guidance for them to overcome the barrier. \\
Likewise, AI of the intelligent tutoring system could be evaluated and determined whether it could qualify as a human teacher. Such system need to negotiate and communicate with a student to get accustomed to newer conditions, and when a student make requests of learning materials or asks question on certain items, the system will adjust automatically to cater to student's need better. In a word, intelligent tutoring system is different from commonplace e-learning websites as it is more flexible and could provide more detailed education contents. This system could store gigantic amount of data, ready to respond to unique needs of students under different scenarios\cite{Jansen2009}.\\
\subsection{Anatomy of the ITS}
Such system has provoked the wide interest amongst researcher and programmer thus many have devoted themselves into designing it, which makes one kind of such system greatly differs from another. It is worthwhile to notice that even these systems have distinctive design theories, they are share the similarities of the following elements: domain, learner, pedagogical and interaction model.\\
Domain model mainly answer the question on how to represent the core knowledge on the computer. It could be demonstrated as flow charts, diagrams, semantic networks, etc. It is mainly consisted of the fundamental logic, strategies and rules to solve ongoing questions. It is the logical core of the system, as it will provide assessment standard when the student's progress is going through evaluation. Moreover, it will also serve as a detector of abnormal behaviors\cite{Ferguson2012}.\\
Learner's model will be demonstrating the systematic evaluation on how well the student is going through one course, what kind of error the student will most likely to make, what kind of learning style the student prefers, what characteristic the student possesses, etc. Such information is collected via students' activities on the system. This kind of model is also utilized in self-regulated learning, which relies little on the help of other human instructors. \\
Pedagogical model focuses more on using best teaching and learning strategies to the student according to the teaching environment. It will check on the student's learning progress, and give out appropriate information or facilitation accordingly.\\
Interaction model, also known as the interactive model are more like a translator between the system and the student. It will need to receive student's input and give out response that could meet the student's needs. Not only this model requires information on the learning material, it would also need information on the common sense of mankind. It was based on verbal texts but nowadays one could identify users' interaction from a variety of sources such as facial expressions, body temperature and moisture, minor gestures, etc.\\
These four models are all under the management of a database. Moreover, the models are designed under the guidance of different educational theories and uprising technologies. As for interactive model, it is based on various multimedia means. To better understand one student's input, NLTK is involved as well as voice recognition software. AIs are introduced to interactive model to automatically output text and voice messages. Capturing technologies are also involved to capture the student's facial expressions and body gestures. Researchers are also trying to bring virtual and augmented reality to the model. This model involves psychology related content as well as researchers are managing to deploy emotional detection technology to determine one's affection state as it might cast profound impact on the learning effect of an individual.\\
When the internet haven't reached today's popularity, many of the ITS were installed on the PC and cannot get frequent upgrade. Due to the hardware limitations of PCs at that time, the function of such ITS is highly limited, as there was little storage space, and PC didn't have high processing speed at that time\cite{Gray1998}. \\
With the rapid development of the internet and PC, ITS have entered an new era as many calculation and data could be processed on the cloud. This have removed the blockade of those who with to be guided by ITS and it is making a growing number of learners benefit from ITS. Nowadays the needed learning materials could be searched and retrieved in no time thanks to ever-developing searching techniques. Moreover, more online wikis are being established which provided supplementary source of domain knowledge to help broaden the borders of domain models.\\
Learners have also witnessed the rapid development of mobile learning in the field of ITS. Wherever there is internet, learner could easily get in touch with ITS at every corner of the world. As smart cell phones and tablets became almost necessities of everyone, ITS have also taken a leap forward and keep absorbing the newest discoveries in such field as machine learning and big data mining. Learners nowadays could get authentic and quick feedback from ITS, which could be a great motivation to the process of learning.\\
\subsection{Designing ITS}
To design an ITS system, researchers have to follow certain procedures, which have certain resemblance with designing a learning management system, or a teaching and learning software.
It is commonly agreed that such process take place in four steps: Needs assessment to carry out the anatomy of learner goals and discussion with the instructor and course material designer for the course; Cognitive task analysis to start building models mentioned before, preparing to tackle any issues in the developmental process; Initial mentor implementation to set up the ecosystem of the ITS and to provide learning facilitation; Evaluation to start trial runs of the ITS and to testify the overall steadiness and robustness of the ITS and give out a holistic assessment to the system.
\subsection{Applying ITS}
One of the most outstanding feature of ITS is that it possesses the capability to give out immediate response to students' needs without a human teacher. Moreover, it can also give timely support, choosing different learning goals for students with different demands, giving individualized coaching and provide mental reinforcement. As a consequence, ITS are more likely to be deployed at institutions, army camps, and business where tutoring and mentoring is required in training yet lacks enough human tutor. Ergo, it have a wide spectrum of application, ranging from kindergarten education, to training on jobs, and even lifelong learning.\\
Researchers have profound interest in studying the efficiency and other benefits of ITS. Such aspects of students as how well they could comprehend the course materials, how eager are they when learning about new contents, how much would they devote themselves into learning and how satisfied they will be after the learning are all taken into consideration. They will even arrange human tutoring session to compare the overall efficiency between human and computer tutors. Some researchers have found out that there is only Little difference between the effect of human tutoring and machine tutoring\cite{Gunn2013}.\\
As the developers have reached the goal of giving response immediately and provide escalated tutoring techniques, they are facing new challenges now. Due to the complexity of such system, ITS is not economy-friendly to design and deploy. As a result, researchers and developers are researching and developing means to make deploying these systems at a lower cost.\\
\subsection{Envisioning ITS}
As mentioned before, the most vital feature of ITS is it does not require extra human tutor to give help to the student. What is more, it could also generate and comprehend natural language for better communication between the machine and the learners. There are many undiscovered areas for researchers to venture in as the recognition rate of the system are still in a moderate level and still have rooms for development. Also, the natural language output give by the ITS sometimes are not considered authentic enough for students to understand. Researchers are also calling for the research on the identification of students' affection state so that the dialogue may change to different mood the student is in accordingly. The system should also be capable of know how the student will be most motivated, thus planning for motivation strategies.\\
The researchers also have the ambition to upgrade the ITS from a system to an environment. It could adapt to more kinds of learners, providing more reliable content and support to learners, and have greater flexibility in the tutoring process. The most advanced ITS could still only function in questions that have clear boundaries and finite solutions. It is all researchers' hope that in the future the ITS will be able to support student with question that have open answers\cite{Hofer2012}.
\section{Conclusion}
For centuries, the learning style of countries around the world remains similar. The teachers are served as the center of information. Students go to school to acquire information they otherwise do not know if they just stay at home. That is the reason why behaviorism was proposed as the main theory of learning. However, with the development of science of technology, people have more tools of getting information such as radio, television and movies. Such development pushed educators to revise their educational strategies in order to make education attractive. Then cognitive theory came into being and provided guidance of how to facilitate the process of learning. However, with the increase of publishing of books and the development of affordable and high speed internet, teachers can no longer serve as people who provide information to students in the fast-changing world. Therefore the strategy of teaching must again change the suit the world. The shift is to foster students ability to solve ill-defined problems through collaboration with minimal guidance from instructors and develop meta-cognitive skills\\
Data mining and recommendation engine have proven their importance nowadays and will continue to shine and make more impact in the foreseeable future in the field of education. The specific field of learning analytic will continue to make more contribution to online learning. However, we must take the ethical use of big data into consideration and make sure we maximize the benefit of big data in education while preventing misusing the data and protect individual's privacy at all costs. Another issue might be the ever-complex algorithms and codes will be a challenge to education specialists and school or learning website admins while the programmer may have limited knowledge in education. However, with the rapid development of educational recommendation system, many new job opportunities will be created, thus encouraging specialists to carry out interdisciplinary research and there will be a growing number of talents that excel both in education theories and programming. It also calls for the tight collaboration between education expert and programmers to make sure that the ever growing education recommendation system backed by big data mining will lead countless of learner to their academic success. The potential of big data on education is still not clear. Although big data have been employed in commerce, healthcare, artificial intelligence and other industries, educators are still waiting to see its implication on learning. However, we can predict big data will bring positive changes to learning as a whole and provide new perspective to instructional design.




\appendix

%Appendix A

\section{Conclusion of roles in the term paper}

In this term paper my partner Weipeng Yang and I participated in the  discussion of the general topic of the paper. We finalized the topic through a meeting. Since we are all students of the Instructional System Technology department at the school of education, we reached an agreement that the topic should be how big data can influence instructional design.\\

Then in the following meetings we had , we generally came up the the structure of the paper. Instructional design is a subject of education. However, instructional design itself is still a borad topic. Therefore, we want to put our focus on Problem-based learning which is an important learning strategy. In addition, we also took our audience into consideration. The potential readers of the paper are not necessarily in the field of instructional design, so we thought it is important to introduce this field of study, the development of instructional design first. And then we will focus on Problem-based learning and give a few examples.\\

And because we are not in the field of big data and this field is really strange to us, we wanted to summarize topics involved in big data and then present how big data will influence instructional design.\\

In this term paper, my responsibility was to focus on the instructional design part and Weipeng was in charge of the big data part. But we also participated in each other's work to keep the group go smoothly. \\









\bibliographystyle{ACM-Reference-Format}

\bibliography{report} 


\end{document}
